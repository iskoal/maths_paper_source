\documentclass[amsthm]{elsart}
\usepackage{amsmath}
\newtheorem{theorem}{Theorem}[section]
\newtheorem{lemma}{Lemma}[section]
\newtheorem{corollary}{Corollary}[section]
\newtheorem{proposition}{Proposition}[section]
\newtheorem{example}{Example}[section]
\newtheorem{remark}{Remark}[section]
\newtheorem*{thmWS}{Theorem WS}
\usepackage{ifpdf}
\usepackage{graphicx,amssymb,lineno}
\usepackage{subfig}
\usepackage{longtable,psfrag}
\usepackage{epstopdf}
\usepackage{float}
\usepackage{cite}
\usepackage{mathrsfs}
\usepackage{latexsym,lineno}
\usepackage{epsfig}
\usepackage{color}
\usepackage{fleqn}
\usepackage{verbatim}\usepackage{epsf}
\usepackage{amsthm}\usepackage{graphicx, float}\usepackage{graphicx}
\usepackage{amsfonts}\usepackage{amssymb}\usepackage{graphpap}
\usepackage{epic}\usepackage{curves}
\ifpdf
\usepackage[%
  pdftitle={Instructions for use of the document class
    elsart},%
  pdfauthor={Simon Pepping},%
  pdfsubject={The preprint document class elsart},%
  pdfkeywords={instructions for use, elsart, document class},%
  pdfstartview=FitH,%
  bookmarks=true,%
  bookmarksopen=true,%
  breaklinks=true,%
  colorlinks=true,%
  linkcolor=blue,anchorcolor=blue,%
  citecolor=blue,filecolor=blue,%
  menucolor=blue,pagecolor=blue,%
  urlcolor=blue]{hyperref}
\else
\usepackage[%
  breaklinks=true,%
  colorlinks=true,%
  linkcolor=blue,anchorcolor=blue,%
  citecolor=blue,filecolor=blue,%
  menucolor=blue,pagecolor=blue,%
  urlcolor=blue]{hyperref}
\fi


%\renewcommand\figurename{}


\renewcommand\floatpagefraction{.2}
\makeatletter
\def\elsartstyle{%
    \def\normalsize{\@setfontsize\normalsize\@xiipt{14.5}}
    \def\small{\@setfontsize\small\@xipt{13.6}}
    \let\footnotesize=\small
    \def\large{\@setfontsize\large\@xivpt{18}}
    \def\Large{\@setfontsize\Large\@xviipt{22}}
    \skip\@mpfootins = 18\p@ \@plus 2\p@
    \normalsize
}
\@ifundefined{square}{}{\let\Box\square}
\makeatother

\def\file#1{\texttt{#1}}

\pagestyle{plain}
\newcounter{Fig}%

\begin{document}

\begin{frontmatter}
\title{The $A_{\alpha}$-spectral radius of a connected graph when a vertex is removed }

\journal{~~}

\author[CW]{Tao She}\ead{she\_tao@163.com},
\author[CW]{Chunxiang Wang*}\ead{wcxiang@mail.ccnu.edu.cn}

\address[CW]{ School of Mathematics and Statistics, Central China Normal University, Wuhan,  P.R. China}

\corauth[cor]{Corresponding author:  chunxiang Wang}




\begin{abstract}
$G$ is a simple connected graph with adjacency matrix $A(G)$ and degree diagonal matrix $D(G)$.
 In 2017, Nikiforov  \cite{2016Merging} defined the matrix $A_{\alpha}(G) = \alpha D(G) + (1- \alpha)A(G)$ for $\alpha \in [0, 1].$
The $A_\alpha$-spectral radius of $G$ is the maximum eigenvalue of $A_\alpha(G)$.
In this paper we give the lower bound of $A_\alpha$-spectral radius when a vertex is removed from $G$.

\vskip 2mm \noindent {\bf Keywords:}  Spectral radius,  Eigenvalue,  Eigenvector,  Adjacency matrix

{\bf AMS subject classification:}


\end{abstract}


\end{frontmatter}

\section{Introduction}

\qquad  We consider simple connected graph $G$ with vertex set $V(G)$ and edge set $E(G)$  throughout this paper.   Let $V(G)=\{v_1, v_2, \cdots, v_k, \cdots, v_n\}$. If any pair of vertices $v_i$ and $v_j$ are adjacent, then we write $v_iv_j \in E(G)$ or $v_i \thicksim v_j$. For a vertex $v_k \in V(G)$, the neighborhood of  $v_k $ is the set $N(v_k) = N_G(v_k) = \{w \in V(G), w  \thicksim  v_k   \}$, and $d_G(v_k)$  denotes the degree of $v_k$ with $ d_{G}(v_k) = |N(v_k)|$.  Let $d_k=d_G(v_k)$  if there
is no ambiguity. For $V_m \subseteq V(G)$ with $|V_m|=m$,   let $G[V_m]$ be the subgraph of $G$ induced by $V_m$, $G - V_m$ be the subgraph induced by $V(G) - V_m$. Let $K_n, K_{s,t}$   denote the clique and complete bipartite graph respectively and  $K_{1,n-1}$ be the star of order $n$.

\qquad  $A(G)$ denotes the adjacency matrix and $D(G)$ denotes the diagonal matrix of the degrees of $G$. The signless Laplacian matrix of $G$ is defined as $Q(G) = D(G)+ A(G).$ Nikiforov \cite{2016Merging} proposed the matrix  $A_{\alpha}(G)$  of a graph $G$ $$A_{\alpha}(G) = \alpha D(G) +(1-\alpha)A(G),$$ for $\alpha \in [0,1]$, which successfully extended the theories of $A(G)$ and $Q(G)$.  Let $A_\alpha$ = $A_\alpha(G)$. It is not hard to see that if $\alpha =0, A_{\alpha}$ is the adjacent matrix, and if $\alpha = \frac{1}{2}$, then $2A_{\frac{1}{2}}$ is the signless Laplacian matrix of $G$.

\par \qquad  Denote  the eigenvalues of $n \times n$ matrix $M$ by $\rho_1(M) \geqslant \rho_2(M) \geqslant \cdots \geqslant \rho_n(M)$. The largest eigenvalue $\rho_\alpha := \rho_1(A_{\alpha}(G))$ of the $A_\alpha$-matrix is defined as the $A_{\alpha}$-spectral radius of $G$.
 Let  $\textbf{x} = (x_1, x_2,x_3, \cdots,x_k, \cdots, x_n)^T$ be a unit nonnegative eigenvector of $\rho_\alpha$ corresponding to  the  vertex set $\{v_1, v_2,v_3$, $\cdots$, $v_k$, $\cdots, v_n\}$.
 Let $G_m =(G - V_m)  \bigcup m K_1$. And $\rho_1(A_\alpha(G_m))$ is called the $A_\alpha$-spectral radius of $G_m$.
The corresponding nonnegative eigenvector of $\rho_1(A_\alpha(G_m))$ is
$\textbf{y}$, normalized such that $\textbf{y}^{T} \textbf{y} = 1$.
For other undefined notations and terminologies, refer to \cite{2001Introduction}.



\qquad Recently, Guo et al. \cite{2019Sharp} and Sun et al. \cite{2019A} presented a relation between $\rho(G)$ and $\rho(G- v_k)$ for adjacency matrices $A(G)$ and $A(G- v_k)$, where $v_k$ is a vertex of $G$. The better bound given by \cite{2019A} is shown as follow $$ \rho(A(G)) \leqslant \sqrt{\rho^2(A(G - v_k)) + 2 d_k - 1}. $$
Though there are many studies of properties of
the $A_\alpha$-matrix in the mathematical literature, we did not find a similar topic of $A_\alpha$-spectral radius of a general connected graph when a vertex is removed.  Naturally, we want to study whether there are similar properties for $A_\alpha(G)$ and $A_\alpha(G-v_k)$. Inspired by the above researches, we obtain the results in this paper.



\section{Main Results}

The theorem of Weyl and So (see, e.g. \cite{1985MatrixAnalysis}, p. 181) is stated below:
\begin{thmWS} \cite{1985MatrixAnalysis} Let $A$ and $B$ be Hermitian matrices of order $n$, and let $1 \leqslant i \leqslant n$ and $1 \leqslant j \leqslant n$.
Then
\begin{eqnarray*}
&\rho _i (A) + \rho _j (B) \leqslant \rho _{i+j-n} (A + B), \quad &if \quad i+j \geqslant n + 1,
\\ &\rho _i (A) + \rho _j (B) \geqslant \rho _{i+j-1} (A + B), \quad &if \quad i+j \leqslant n + 1.
\end{eqnarray*}
In either of these inequalities, equality holds if and only if there exists a nonzero $n$-vector that is an eigenvector to each of the three eigenvalues involved.
\end{thmWS}

From the following proposition given by Nikiforov, we know that $\textbf{x}$ is a positive eigenvector corresponding to $\rho _\alpha$ if $\alpha \in [0, 1)$.
\begin{proposition}[Proposition 13 of \cite{2016Merging}] \label{prop:perron}
Let $\alpha \in [0, 1)$, let $G$ be a connected graph, and let $\textbf{x}$ be a nonnegative eigenvector to $\rho( A_\alpha(G))$, then $\textbf{x}$ is positive and is unique up to scaling.
\end{proposition}

The $A_\alpha$-spectrum of $K_n$ and $K_{1,n-1}$ are given by Nikiforov as follows:
\begin{proposition}[Proposition 36 of \cite{2016Merging}] \label{prop:3}
The eigenvalues of $A_\alpha(K_n)$  are
\begin{eqnarray*}
\rho_1 (A_\alpha(K_n)) &&= n - 1 ,
\\ \rho _k (A_\alpha(K_n)) &&= \alpha n - 1 \quad for \quad 1 < k \leqslant n .
\end{eqnarray*}
\end{proposition}

\begin{proposition} [Proposition 38 of \cite{2016Merging}] \label{prop:1}
The eigenvalues of $A_\alpha(K_{1,n-1})$  are
\begin{eqnarray*}
\rho_1 (A_\alpha(K_{1,n-1})) &&= \frac{1}{2} \big( \alpha n + \sqrt{\alpha ^2 n^2 + 4(n-1)(1-2\alpha)} \, \big)
\\ \rho _n (A_\alpha(K_{1,n-1})) &&= \frac{1}{2} \big( \alpha n - \sqrt{\alpha ^2 n^2 + 4(n-1)(1-2\alpha)} \, \big)
\\ \rho _k (A_\alpha(K_{1,n-1})) &&= \alpha \quad for \quad 1<k<n.
\end{eqnarray*}
\end{proposition}

Using Theorem WS, we get the relation between $\rho _\alpha (G - v_k)$ and $\rho _\alpha (G)$.
\begin{theorem} \label{the:0}
Let $\alpha \in [0, 1)$, $v_k$ is a vertex of a connected graph $G$ with degree $d_k$, then
\begin{eqnarray*}
\rho _\alpha (G - v_k) \geqslant \rho _\alpha (G) - \frac{1}{2} \big( \alpha (d_k + 1) + \sqrt{\alpha ^2 (d_k + 1)^2 + 4 d_k (1-2\alpha)} \, \big)
\end{eqnarray*}
with equality holds if and only if $G \cong K_{1,n-1}$ and $d_k = n - 1$.
\end{theorem}
\begin{proof} Let $E_k $ denote the set of incident edges of $v_k$.
 \\ Let $G_1 = G - E_k$, $G_1' = G - E(G_1)$,
 then $G_1$ and $G_1'$ are spanning subgraphs of $G$, with
 $G = G_1 \bigcup G_1'$ ,
 $|E(G)| = |E(G_1)| + |E(G_1')|$.
\\ It is easy to see that
\begin{eqnarray*}
 A(G_1) + A(G_1') \quad &&= A(G),
\\  A_\alpha (G_1) + A_\alpha (G_1') &&= A_\alpha (G).
\end{eqnarray*}
 Since graph $G_1'$ is the disjoint union of $K_{1,d_k}$ and isolated vertices, by Proposition \ref{prop:1} we get $\rho _\alpha (G_1') = \rho _\alpha (K_{1,d_k}) = \frac{1}{2} \big( \alpha (d_k + 1) + \sqrt{\alpha ^2 (d_k + 1)^2 + 4 d_k (1-2\alpha)} \, \big). $
\\ Theorem WS implies that $\rho _\alpha (G_1) + \rho _\alpha (G_1') \geqslant \rho _\alpha (G)$.
\\ Hence
\begin{eqnarray*}
\rho _\alpha (G - v_k) &&= \rho _\alpha (G_1)
\\ && \geqslant \rho _\alpha (G) - \rho _\alpha (G_1')
\\ && = \rho _\alpha (G) - \frac{1}{2} \big( \alpha (d_k + 1) + \sqrt{\alpha ^2 (d_k + 1)^2 + 4 d_k (1-2\alpha)} \, \big).
\end{eqnarray*}

By Theorem WS, equality holds if and only if there exists a nonzero $n$-vector $\textbf{z} = (z_1, z_2,z_3, \cdots,z_k, \cdots, z_n)^T$ that is an eigenvector to each of $\rho _\alpha (G)$, $\rho _\alpha (G_1)$ and $\rho _\alpha (G_1')$. Suppose   $a_{ij}^{\prime}$ is the ($i$)th row ($j$)th column element of $A_\alpha(G_m)$, then
\begin{eqnarray} \label{matrix:G1}
A_\alpha(G_1) \;\;=\;\;
\begin{pmatrix}
a_{11}^{\prime} & \cdots & a_{1(k-1)}^{\prime} & 0 & a_{1(k+1)}^{\prime} & \cdots & a_{1n}^{\prime} \\
\vdots & {} & \vdots & \vdots & \vdots & {} & \vdots \\
a_{(k-1)1}^{\prime} & \cdots & a_{(k-1)(k-1)}^{\prime} & 0 & a_{(k-1)(k+1)}^{\prime} & \cdots & a_{(k-1)n}^{\prime} \\
0 & \cdots & 0 & 0 & 0 & \cdots & 0 \\
a_{(k+1)1}^{\prime} & \cdots & a_{(k+1)(k-1)}^{\prime} & 0 & a_{(k+1)(k+1)}^{\prime} & \cdots & a_{(k+1)n}^{\prime} \\
\vdots & {} & \vdots & \vdots & \vdots & {} & \vdots \\
a_{n1}^{\prime} & \cdots & a_{n(k-1)}^{\prime} & 0 & a_{n(k+1)}^{\prime} & \cdots & a_{nn}^{\prime}
\end{pmatrix}
\end{eqnarray}
Since $\rho _\alpha (G_1) \textbf{z}$ = $A_\alpha (G_1) \textbf{z}$ \  and \ $A_\alpha (G_1)$ is a matrix as above,
 therefore  $\rho _\alpha (G_1) z_k = 0$. By proposition \ref{prop:perron}, $\textbf{z}$ is a multiple of a positive eigenvector corresponding to $\rho _\alpha (G)$. Thus $z_k \neq 0$ and $\rho _\alpha (G_1) = 0$. Hence $G_1$ must be a zero matrix. Equality holds if and only if $G \cong K_{1,n-1}$ and $d_k = n - 1$.
\end{proof}


\begin{lemma} \label{lem:1}
Let $\alpha \in [0, 1)$, $v_k$ is a vertex of a connected graph $G$ with degree $d_k$, then $\rho _\alpha > \alpha d_k$.
\end{lemma}
\begin{proof}
$x_i$ is the $i$th component of the positive eigenvector $\textbf{x}$ corresponding to $\rho_\alpha$. Since $x_k > 0$, $x_i > 0$, $1-\alpha > 0$ and
\begin{eqnarray*}
\rho _\alpha x_k = \sum \limits_{i = 1}^{n} a_{ki} x_i = (1 - \alpha) \sum \limits_{v_i \in N(v_k)}^{} x_i + \alpha d_k x_k ,
\end{eqnarray*}
the result is obvious.
\end{proof}

\begin{lemma} \label{lem:2.2}
Let $\alpha \in [0, 1)$, $V_m \subseteq V(G)$ with $|V_m|=m$ and $ c_s = |N_G(v_s) \bigcap V_m|$ for $v_s\notin V_m$.
Suppose $a_{ij}$ is the ($i$)th row ($j$)th column element of $A_\alpha$, and $x_i$ is the $i$th component of the positive unit eigenvector $\textbf{x}$ corresponding to $\rho_\alpha$.Then
\begin{equation} \label{equ:11}
\sum \limits_{v_k \in V_m}^{} x_k^2
\leqslant \frac{1}{2} + \frac{1}{2 \rho_\alpha} \Big( \sum \limits_{v_i \in V_m}^{} \sum \limits_{v_j \in V_m}^{} a_{ij} x_i x_j - \alpha \sum \limits_{v_s \notin V_m}^{} c_s x_s^2 \Big)
\end{equation}
with equality holds if and only if $G_m$ is an empty graph.
\end{lemma}
\begin{proof}
Since $A_\alpha = \alpha D(G) + (1 - \alpha)A(G)$, we have
\begin{eqnarray*}
 a_{ij} =
\left\{
\begin{aligned}
&1 - \alpha    && if \quad i \neq j  \quad and \quad v_i \in N_G(v_j) \\
&0    && if \quad i \neq j  \quad and \quad v_i \notin N_G(v_j) \\
&\alpha d_i    && if \quad i = j
\end{aligned}
\right.
\end{eqnarray*}
Suppose   $a_{ij}^{\prime}$ is the ($i$)th row ($j$)th column element of $A_\alpha(G_m)$, then
\begin{eqnarray*}
 a_{ij}^{\prime} =
\left\{
\begin{aligned}
&0    && if \quad v_i \in V_m \quad or \quad v_j \in V_m \\
&a_{ij}    && if \quad i \neq j  \quad and \quad v_i \notin V_m  \quad and \quad v_j \notin V_m \\
&a_{ii} - \alpha c_i    && if \quad i = j \quad and \quad v_i \notin V_m
\end{aligned}
\right.
\end{eqnarray*}
Let $\textbf{a}_k$ be the column vector $(a_{k1}, a_{k2}, \cdots , a_{kn})^T$ and $\textbf{e}_k$ be the $k$th basis column vector $(0, 0, \cdots , 0, 1, 0, \cdots , 0)^T$,
 where only the $k$th component is 1. We have
\begin{eqnarray*}
 && \quad \textbf{x}^T(A_\alpha - A_\alpha(G_m))\textbf{x}
\\ &&= \textbf{x}^T \Big(\sum \limits_{v_k \in V_m}^{} (\textbf{a}_k \textbf{e}_k^T + \textbf{e}_k \textbf{a}_k^T)
- \sum \limits_{v_i \in V_m}^{} \sum \limits_{v_j \in V_m}^{} a_{ij} \textbf{e}_i \textbf{e}_j^T \Big) \textbf{x}
+ \sum \limits_{v_s \notin V_m}^{} \alpha c_s x_s^2
\\ &&= \sum \limits_{v_k \in V_m}^{} (\textbf{x}^T \textbf{a}_k \textbf{e}_k^T \textbf{x} + \textbf{x}^T \textbf{e}_k \textbf{a}_k^T \textbf{x}) - \sum \limits_{v_i \in V_m}^{} \sum \limits_{v_j \in V_m}^{} a_{ij} \textbf{x}^T \textbf{e}_i \textbf{e}_j^T \textbf{x} + \sum \limits_{v_s \notin V_m}^{} \alpha c_s x_s^2 \\
\\ &&= 2x_k \sum \limits_{i=1}^{n}x_ia_{ki} - \sum \limits_{v_i \in V_m}^{} \sum \limits_{v_j \in V_m}^{} a_{ij} x_i x_j
   + \alpha \sum \limits_{v_s \notin V_m}^{} c_s x_s^2
\\ &&=  2 \rho_\alpha \sum \limits_{v_k \in V_m}^{} x_k^2 - \sum \limits_{v_i \in V_m}^{} \sum \limits_{v_j \in V_m}^{} a_{ij} x_i x_j
   + \alpha \sum \limits_{v_s \notin V_m}^{} c_s x_s^2
\end{eqnarray*}
Hence
\begin{eqnarray*}
\textbf{x}^T A_\alpha(G_m) \textbf{x}
&&= \textbf{x}^T A_\alpha \textbf{x} - \textbf{x}^T(A_\alpha - A_\alpha(G_m))\textbf{x}
\\ &&= \rho_\alpha
   - 2 \rho_\alpha \sum \limits_{v_k \in V_m}^{} x_k^2
   + \sum \limits_{v_i \in V_m}^{} \sum \limits_{v_j \in V_m}^{} a_{ij} x_i x_j
   - \alpha \sum \limits_{v_s \notin V_m}^{} c_s x_s^2
\\ &&\geqslant 0
\end{eqnarray*}
Which implies
\begin{eqnarray*}
\sum \limits_{v_k \in V_m}^{} x_k^2
\leqslant \frac{1}{2} + \frac{1}{2 \rho_\alpha} \Big( \sum \limits_{v_i \in V_m}^{} \sum \limits_{v_j \in V_m}^{} a_{ij} x_i x_j - \alpha \sum \limits_{v_s \notin V_m}^{} c_s x_s^2 \Big)
\end{eqnarray*}
 Note that $\textbf{x}$ is a positive unit eigenvector, we can see that the equality holds if and only if $G_m$ is an empty graph.
\end{proof}

\begin{lemma} \label{lem:2}
Let $\alpha \in [0, 1)$, $\textbf{x}$ is the positive eigenvector of $A_\alpha (G)$ corresponding to $\rho _\alpha$, with $\textbf{x}^T \textbf{x} = 1$. $x_k$ is the $k$th component of $\textbf{x}$, $v_k$ is  the $k$th vertex of $G$. Then
\begin{equation} \label{equ:13}
\rho _\alpha (G - v_k)
    \geqslant \rho _\alpha - \alpha - \frac{x_k^2}{1 - x_k^2} (\rho _\alpha - \alpha d_k)
\end{equation}
\end{lemma}

\begin{proof}
Let $V_m = V(G) - v_k$, then $G_m$ is an empty graph. By Lemma \ref{lem:2.2}, we have
\begin{eqnarray*}
\sum \limits_{v_i \in V_m}^{} x_i^2
= \frac{1}{2} + \frac{1}{2 \rho_\alpha} \Big( \sum \limits_{v_i \in V_m}^{} \sum \limits_{v_j \in V_m}^{} a_{ij} x_i x_j - \alpha \sum \limits_{v_s \notin V_m}^{} c_s x_s^2 \Big)
\end{eqnarray*}
Let $\hat{\textbf{x}} = \{x_1, \cdots, x_{k-1}, 0, x_{k+1}, \cdots, x_n\}^T$ , whose $k$th component is 0. \\
We get
\begin{eqnarray*}
1 - x_k^2
&&= \sum \limits_{v_i \in V_m}^{} x_i^2
\\ &&= \frac{1}{2} + \frac{1}{2 \rho_\alpha} \Big( \sum \limits_{v_i \in V_m}^{} \sum \limits_{v_j \in V_m}^{} a_{ij} x_i x_j - \alpha d_k x_k^2 \Big)
\\ &&= \frac{1}{2} + \frac{1}{2 \rho_\alpha} \Big( \textbf{x}^T A_\alpha(G_1) \textbf{x} + \alpha \sum \limits_{v_i \in N(v_k)}^{} x_i^2 - \alpha d_k x_k^2 \Big)
\\ &&= \frac{1}{2} + \frac{1}{2 \rho_\alpha} \Big( \hat{\textbf{x}}^T A_\alpha(G_1) \hat{\textbf{x}} + \alpha \sum \limits_{v_i \in N(v_k)}^{} x_i^2 - \alpha d_k x_k^2 \Big)
\\ &&\leqslant \frac{1}{2} + \frac{1}{2 \rho_\alpha} \Big( \rho_\alpha (G_1) \hat{\textbf{x}}^T \hat{\textbf{x}} + \alpha - \alpha  x_k^2 - \alpha d_k x_k^2 \Big)
\\ &&= \frac{1}{2} + \frac{1}{2 \rho_\alpha} \Big( \rho_\alpha (G - v_k) (1 - x_k^2) + \alpha - \alpha  x_k^2 - \alpha d_k x_k^2 \Big)
\end{eqnarray*}
Which implies
\begin{eqnarray*}
  \rho_\alpha (G - v_k) \geqslant \rho_\alpha - \alpha - \frac{x_k^2}{1 - x_k^2} ( \rho_\alpha - \alpha d_k).
\end{eqnarray*}
\end{proof}

\begin{lemma} \label{lem:3}
Let $\alpha \in [0, 1)$, $\textbf{x}$ is the positive eigenvector of $A_\alpha (G)$ corresponding to $\rho _\alpha$  with $\textbf{x}^T \textbf{x} = 1$, $x_k$ is the $k$th component of $\textbf{x}$. Then \\
\begin{equation} \label{equ:14}
x_k^2 \leqslant \frac{(1 - \alpha)^2 d_k}{(\rho _\alpha - \alpha d_k)^2 + (1 - \alpha)^2 d_k}
\end{equation}
\end{lemma}

\begin{proof}
Since
\begin{eqnarray*}
\rho _\alpha x_k = \sum \limits_{i = 1}^{n} a_{ki} x_i = (1 - \alpha) \sum \limits_{v_i \in N(v_k)}^{} x_i + \alpha d_k x_k
\end{eqnarray*}
We have
\begin{eqnarray*}
 (\rho _\alpha - \alpha d_k)^2 x_k^2
    &&= (1 - \alpha)^2 \big(\sum \limits_{v_i \in N(v_k)}^{} x_i \big)^2
    \\ &&\leqslant  (1 - \alpha)^2 d_k \sum \limits_{v_i \in N(v_k)}^{} x_i^2
    \\ &&\leqslant  (1 - \alpha)^2 d_k (1 - x_k^2)
\end{eqnarray*}
That is
\begin{eqnarray*}
x_k^2 \leqslant \frac{(1 - \alpha)^2 d_k}{(\rho _\alpha - \alpha d_k)^2 + (1 - \alpha)^2 d_k}
\end{eqnarray*}
\end{proof}

\begin{theorem} \label{the:4}
Let $\alpha \in [0, 1)$, $v_k$ is a vertex of a connected graph $G$ with degree $d_k$.  Then
\begin{equation} \label{equ:16}
\rho _\alpha (G - v_k)
  \geqslant \rho _\alpha - \alpha - \frac{(1 - \alpha)^2 d_k}{\rho _\alpha - \alpha d_k}
\end{equation}
with equality holds if and only if $G$ is the join of the vertex $v_k$ and a regular graph on $n-1$ vertices.
\end{theorem}

\begin{proof}
$\textbf{x}$ is the positive eigenvector of $A_\alpha (G)$ corresponding to $\rho _\alpha$  with $\textbf{x}^T \textbf{x} = 1$, $x_k$ is the $k$th component of $\textbf{x}$.
By Lemma \ref {lem:2}, we obtain \\
\begin{eqnarray} \label{equ:21}
\rho _\alpha (G - v_k)
    \geqslant \rho _\alpha - \alpha - \frac{x_k^2}{1 - x_k^2} (\rho _\alpha - \alpha d_k)
\end{eqnarray}
By Lemma \ref {lem:3} we get
\begin{eqnarray} \label{equ:22}
x_k^2 \leqslant \frac{(1 - \alpha)^2 d_k}{(\rho _\alpha - \alpha d_k)^2 + (1 - \alpha)^2 d_k}.
\end{eqnarray}
\\
Since $h(x_k) = \frac{x_k^2}{1 - x_k^2} $ is a increasing function on $x_k \in (0, 1)$, $\rho _\alpha - \alpha d_k > 0$ if $\alpha \in [0, 1)$ (by lemma \ref{lem:1}), combining (\ref{equ:21})with (\ref{equ:22}), we have  \\
\begin{eqnarray*}
\rho _\alpha (G - v_k)
  &&\geqslant \rho _\alpha - \alpha - \frac{x_k^2}{1 - x_k^2} (\rho _\alpha - \alpha d_k)
  \\ &&\geqslant \rho _\alpha - \alpha - \frac{(1 - \alpha)^2 d_k}{(\rho _\alpha - \alpha d_k)^2} (\rho _\alpha - \alpha d_k)  \\
  \\ &&\geqslant \rho _\alpha - \alpha - \frac{(1 - \alpha)^2 d_k}{\rho _\alpha - \alpha d_k}
.
\end{eqnarray*}
This completes the proof of inequality (\ref{equ:16}).
By the proof of Lemma \ref {lem:2} and Lemma \ref {lem:3}, we know that the equality holds if and only if $G$ is the join of the vertex $v_k$ and a regular graph on $n-1$ vertices.
\end{proof}

\begin{corollary} \label{coro:1}
Let $G$ be the join of the vertex $v_k$ and a $r$-regular graph on $n-1$ vertices, $\alpha \in [0, 1)$.  Then
\begin{eqnarray*}
&&(i) \quad \rho _\alpha = \frac{1}{2}(\alpha n + r) + \frac{1}{2} \sqrt{(\alpha n + r)^2 - 4 (r \alpha + 2 \alpha - 1) ( n - 1) }.
\\ &&(ii) \quad \rho _1(A(G)) = \frac{r}{2} + \frac{1}{2} \sqrt{r ^2 + 4 ( n - 1) }.
\\ &&(iii) \quad \rho _1(Q(G)) = \frac{n}{2} + r +  \sqrt{(\frac{n}{2} + r)^2 - 2 r ( n - 1) }.
\end{eqnarray*}
\end{corollary}

\begin{proof}
$(i) \quad$ Since $G - v_k$ is a $r$-regular graph, it is obvious that $\rho _\alpha (G - v_k) = r$.
By Theorem \ref{the:4}, we have
\begin{eqnarray*}
r = \rho _\alpha (G - v_k)
  = \rho _\alpha - \alpha - \frac{(1 - \alpha)^2 d_k}{\rho _\alpha - \alpha d_k}
  = \rho _\alpha - \alpha - \frac{(1 - \alpha)^2 (n - 1)}{\rho _\alpha - \alpha (n - 1)}.
\end{eqnarray*}
Hence
\begin{eqnarray*}
&& \rho _\alpha ^2 - (\alpha n + r) \rho _\alpha + (r \alpha + 2 \alpha - 1) ( n - 1) = 0.
\\ && \rho _\alpha = \frac{1}{2}(\alpha n + r) + \frac{1}{2} \sqrt{(\alpha n + r)^2 - 4 (r \alpha + 2 \alpha - 1) ( n - 1) }.
\end{eqnarray*}
\begin{eqnarray*}
&&(ii) \quad \rho _1(A(G)) = \rho _\alpha | _{\alpha = 0} = \frac{r}{2} + \frac{1}{2} \sqrt{r ^2 + 4 ( n - 1) } .
\\ &&(iii) \quad \rho _1(Q(G)) = 2 \rho _\alpha | _{\alpha = \frac{1}{2}} = \frac{r}{2} + \frac{1}{2} \sqrt{r ^2 + 4 ( n - 1) } .
\end{eqnarray*}
\end{proof}

\begin{remark}
By inequality (\ref{equ:16}) and proposition \ref{prop:3}, we get
\begin{eqnarray*}
n - 2 &&= \rho_\alpha(K_{n - 1})
\\ &&\geqslant \rho _\alpha(K_n) - \alpha - \frac{(1 - \alpha)^2 d_k}{\rho _\alpha(K_n) - \alpha d_k}
\\ &&= (n-1) - \alpha - \frac{(1 - \alpha)^2 (n-1)}{(n-1) - \alpha (n-1)}
\\ &&= n - 2
\end{eqnarray*}
So the equality in (\ref{equ:16})  holds if $G \cong K_n$.
\end{remark}

By Theorem \ref{the:4}, let $\alpha = 0$ , we have the following corollary.
\begin{corollary} \label{coro:4}
Let $v_k$ be a vertex of a connected graph $G$ with degree $d_k$.  Then
\begin{equation} \label{equ:coro4}
\rho _1 (A(G - v_k))
  \geqslant \rho _1(A(G)) - \frac{d_k}{\rho _1(A(G))}
\end{equation}
with equality holds if and only if $G$ is the join of the vertex $v_k$ and a regular graph on $n-1$ vertices.
\end{corollary}

By Theorem \ref{the:4}, let $\alpha = \frac{1}{2}$, we have the following corollary.
\begin{corollary} \label{coro:5}
Let $v_k$ be a vertex of a connected graph $G$ with degree $d_k$. $Q(G)$  is the signless Laplacian matrix of $G$. Then
\begin{equation} \label{equ:coro5}
\rho _1 (Q(G - v_k))
  \geqslant \rho _1(Q(G)) - 1 - \frac{d_k}{\rho _1(Q(G)) - d_k}
\end{equation}
with equality holds if and only if $G$ is the join of the vertex $v_k$ and a regular graph on $n-1$ vertices.
\end{corollary}


\begin{proposition} [Corollary 1 of \cite{2019Sharp}] \label{prop:remark1}
Let $G$ be a connected graph with $n$ vertices. Then
$$ \rho_1(A(G)) \geqslant max \{ \sqrt{d(v)},  v \in V(G)  \}, $$
with equality holds if and only if $G = K_{1,n-1}$.
\end{proposition}

\begin{remark} \label{remark:1}
By Proposition \ref{prop:remark1}, we have
\begin{equation*}
\rho _1(A(G)) - \frac{d_k}{\rho _1(A(G))} > 0.
\end{equation*}
By Corollary \ref{coro:4}, we have
\begin{eqnarray*}
\rho _1 (A(G - v_k))
&& \geqslant \rho _1(A(G)) - \frac{d_k}{\rho _1(A(G))}.
\\ && = \sqrt{\rho _1^2(A(G)) - 2d_k + \frac{d_k^2}{\rho _1^2(A(G))}}.
\end{eqnarray*}
Hence
\begin{equation*}
\rho _1(A(G)) \leqslant \sqrt{\rho _1^2 (A(G - v_k)) + 2 d_k - \frac{d_k^2}{\rho _1^2(A(G))}}.
\end{equation*}
 Compare with the bound \cite{2019A} mentioned in the introduction section
 \begin{equation} \label{equ:bound1}
 \rho(A(G)) \leqslant \sqrt{\rho^2(A(G - v_k)) + 2 d_k - 1},
 \end{equation}
 we can see that Corollary \ref{coro:4} is better than (\ref{equ:bound1}) if $ d_k > \rho _1(A(G))$.
\end{remark}


\begin{remark} \label{remark:2}

Theorem \ref{the:0} and Theorem \ref{the:4} study the lower bound of $\rho_\alpha(G-v_k)$, with expression of $\rho_\alpha$, $\alpha$ and $d_k$. Theorem \ref{the:4} is better than Theorem \ref{the:0} for $\alpha = 0$ or $\alpha = \frac{1}{2}$. The lower bound of $\rho_\alpha(G-V_m)$ can be enhanced.
\end{remark}

\vskip4mm\noindent{\bf Acknowledgements.}


 The work was partially supported by the National Natural Science Foundation of China under Grants 11771172,12061039.




\begin{thebibliography}{99}

\bibitem{2016Merging}
\textcolor{blue}{
V. Nikiforov.
\newblock Merging the $A$-and $Q$-spectral theories.
\newblock {Applicable Analysis \& Discrete Mathematics}, 11(1) (2017), 81-107.}

\bibitem{2017SpectraOfTrees}
\textcolor{blue}{
V. Nikiforov, G. Past\'{e}n, O. Rojo and R.L. Soto.
\newblock On the $A_\alpha$-spectra of trees.
\newblock {Linear Algebra and its Applications}, 520 (2017), 286-305.}

\bibitem{2017PositiveSemidefinitness}
\textcolor{blue}{
V. Nikiforov and O. Rojo.
\newblock A note on the positive semidefinitness of $A_\alpha(G)$.
\newblock {Linear Algebra and its Applications}, 519 (2017), 156-163.}

\bibitem{2018PendentPaths}
\textcolor{blue}{
V. Nikiforov and O. Rojo.
\newblock On the $\alpha$-index of graphs with pendent paths.
\newblock {Linear Algebra and its Applications}, 550 (2018), 87-104.}

\bibitem{2019Sharp}
\textcolor{blue}{
J.M. Guo, Z.W. Wang, X. Li.
\newblock Sharp upper bounds of the spectral radius of a graph
\newblock {Discrete Mathematics}, 2019, 342(9):2559-2563.}

\bibitem{2019A}
\textcolor{blue}{
S.~Sun and K.~C. Das.
\newblock A conjecture on the spectral radius of graphs.
\newblock {Linear Algebra and its Applications}, 2019, 588.}

\bibitem{2012Bounds}
\textcolor{blue}{
L.~Cong, H.~Wang, and P.~V. Mieghem.
\newblock Bounds for the spectral radius of a graph when nodes are removed.
\newblock {Linear Algebra and its Applications}, 2012, 437(1):319-323.}


\bibitem{1985MatrixAnalysis}
\textcolor{blue}{
R. Horn and C. Johnson.
\newblock Matrix Analysis,
\newblock {Cambridge University Press, Cambridge}, 1985, xiii+561 pp.}

\bibitem{2001Introduction}
\textcolor{blue}{
DB West.
\newblock Introduction to graph theory - second edition.
\newblock 2001.}

%\bibitem{2016Merging}
%\textcolor{blue}{
%V.~Nikiforov.
%\newblock Merging the a- and q-spectral theories.
%\newblock {\em Applicable Analysis \& Discrete Mathematics}, 11(1), 2016.}









\end{thebibliography}

\end{document}
