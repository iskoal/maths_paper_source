\documentclass[amsthm]{elsart}
\usepackage{amsmath}
\newtheorem{theorem}{Theorem}[section]
\newtheorem{lemma}{Lemma}[section]
\usepackage{ifpdf}
\usepackage{graphicx,amssymb,lineno}
\usepackage{subfig}
\usepackage{longtable,psfrag}
\usepackage{epstopdf}
\usepackage{float}
\usepackage{cite}
\usepackage{mathrsfs}
\usepackage{latexsym,lineno}
\usepackage{epsfig}
\usepackage{color}
\usepackage{fleqn}\usepackage{verbatim}\usepackage{epsf}
\usepackage{amsthm}\usepackage{graphicx, float}\usepackage{graphicx}
\usepackage{amsfonts}\usepackage{amssymb}\usepackage{graphpap}
\usepackage{epic}\usepackage{curves}
\ifpdf
\usepackage[%
  pdftitle={Instructions for use of the document class
    elsart},%
  pdfauthor={Simon Pepping},%
  pdfsubject={The preprint document class elsart},%
  pdfkeywords={instructions for use, elsart, document class},%
  pdfstartview=FitH,%
  bookmarks=true,%
  bookmarksopen=true,%
  breaklinks=true,%
  colorlinks=true,%
  linkcolor=blue,anchorcolor=blue,%
  citecolor=blue,filecolor=blue,%
  menucolor=blue,pagecolor=blue,%
  urlcolor=blue]{hyperref}
\else
\usepackage[%
  breaklinks=true,%
  colorlinks=true,%
  linkcolor=blue,anchorcolor=blue,%
  citecolor=blue,filecolor=blue,%
  menucolor=blue,pagecolor=blue,%
  urlcolor=blue]{hyperref}
\fi


%\renewcommand\figurename{}


\renewcommand\floatpagefraction{.2}
\makeatletter
\def\elsartstyle{%
    \def\normalsize{\@setfontsize\normalsize\@xiipt{14.5}}
    \def\small{\@setfontsize\small\@xipt{13.6}}
    \let\footnotesize=\small
    \def\large{\@setfontsize\large\@xivpt{18}}
    \def\Large{\@setfontsize\Large\@xviipt{22}}
    \skip\@mpfootins = 18\p@ \@plus 2\p@
    \normalsize
}
\@ifundefined{square}{}{\let\Box\square}
\makeatother

\def\file#1{\texttt{#1}}

\pagestyle{plain}
\newcounter{Fig}%

\begin{document}

\begin{frontmatter}
\title{Total Domination Colorings and Total Domination in Trees }

\journal{~~}

 \author[CW]{Tao She}\ead{28810707@qq.com},
\author[CW]{Chunxiang Wang}\ead{wcxiang@mail.ccnu.edu.cn},



\address[CW]{ School of Mathematics and Statistics, Central China Normal University, Wuhan,  P.R. China}



\corauth[cor]{Corresponding author:  chunxiang Wang}




\begin{abstract}



\vskip 2mm \noindent {\bf Keywords:}
{\bf AMS subject classification:}



\end{abstract}


\end{frontmatter}

\section{Introduction}

% The graphs considered in this paper are finite, undirected, and simple (no loops or multiple edges). Let $G= (V(G), E(G))$ be a
%graph with vertex set $V(G)=\{v_1, v_2, \cdots, v_n\}$ and edge set $E(G)=\{e_1,e_2,\cdots,e_m\}$. $|V(G)|=n$ is called the order of graph $G$, and $|E(G)|=m$ is called the size of graph $G$.  The open neighborhood of $v$ is $N_G(v)=\{u| uv\in E(G)\}$. The degree of a vertex $v$ in $G$ is   $d_G(v)=|N_G(v)|$. A path $P(v_0,v_k)$ in a graph is a sequence of vertices and edges $v_0, e_1, v_1, \cdots, e_k, v_k$, in which each edge $e_i=v_{i-1}v_i$ and no vertex is repeated.

\qquad Let G be a graph with adjacency matrix A(G), and D(G) be the diagonal matrix of its vertex degrees. In  \cite{2016Merging}, it was proposed to study the family of matrices $A_\alpha(G)$ defined for any real $\alpha\in[0,1]$ as
$A_\alpha(G) = \alpha D(G) + (1 - \alpha)A(G)$.
And
$\rho_1(A_\alpha(G))$ is called the $A_\alpha$-spectral radius of G. The corresponding eigenvector of $\rho_1(A_\alpha(G))$ is
x, normalized such that $x^{T}
x = 1$.
 Let $A_\alpha$ = $A_\alpha(G)$, $\rho_\alpha$ = $\rho_1(A_\alpha(G))$.
\par \qquad 
Let $V_m$ denote the subset
of V(G) which contains m vertices, $V_m=\{v_{i_1}, v_{i_2}, \cdots, v_{i_n}\}$. For $v_k \in$ V(G), the degree $d_G(v_k)$ of $v_k$ is the number of vertices which are adjacent to $v_k$ in G. We write $d_k$ for $d_G(v_k)$ if there
is no ambiguity.
$G_m = (G - V_m) \bigcup V_m$ is a subgraph from G by changing $V_m$ to be an independent set of G.
And
$\rho_1(A_\alpha(G_m))$ is called the $A_\alpha$-spectral radius of $G_m$. 
The corresponding eigenvector of $\rho_1(A_\alpha(G_m))$ is
$\omega$, normalized such that $\omega^{T}
\omega = 1$.

\begin{theorem}
For any graph G and corresponding graph $G - V_m$, obtained from G by removing
the set $V_m$ of m nodes, it holds that

\begin{equation} %\label{(1)}
(1 - 2\sum \limits_{v_i \in V_m}^{} x_i^{_2})\rho_\alpha
+ \sum \limits_{v_i \in V_m}^{} \sum \limits_{v_j \in V_m}^{} a_{ij}x_ix_j
- \alpha m \sum \limits_{v_k \notin V_m}^{} x_k^{_2} \\
\leq \rho_\alpha(G - V_m)  
\leq  \rho_\alpha.
\end{equation}

%\sum \limits_{i=1}^n w_if_i
where x is the eigenvector of $A_\alpha$ corresponding to the largest eigenvalue $\rho_\alpha$. In particular, if m = 1, then

\begin{equation}
(1 - 2x_k^{_2})\rho_\alpha
+ \alpha (x_k^{_2} + d_k x_k^{_2} - 1)  \leq
\rho_\alpha(G - v_k)  \leq  \rho_\alpha.
\end{equation}

\end{theorem}
\textbf{Proof.}
Suppose $a_{ij}$ is the (i)th row (j)th column element of $A_\alpha$,  $a_{ij}^{\prime}$ is the (i)th row (j)th column element of $A_\alpha(G_m)$, let $N(v_i)$ be the neighbor set of $v_i$, then \\
 $a_{ij}^{\prime} = 
\left\{
\begin{aligned}
&a_{ij}    && if \quad i \neq j \\
&a_{ii} - \alpha c_i    && if \quad i = j \quad and \quad v_i \notin V_m   \\
&0    && if \quad i = j \quad and \quad v_i \in V_m \\
\end{aligned}
\right.
\\ 
where \quad c_i = |N(v_i) \bigcap V_m|,  %d_{G[v_i \bigcup V_m]}(v_i), 
\quad 0\leq c_i \leq m. 
$ \\
After removing a node $v_k$ from graph G, we obtain $A_\alpha(G - v_k)$
, which is a (n - 1) $\times$ (n - 1) matrix, \\
$A_\alpha(G - v_k) =
\begin{pmatrix}
a_{11}^{\prime} & \cdots & a_{1(k-1)} & a_{1(k+1)} & \cdots & a_{1n}\\
\vdots & {} & \vdots \vdots & {} & \vdots \\
a_{(k-1)1} & \cdots & a_{(k-1)(k-1)}^{\prime} & a_{(k-1)(k+1)} & \cdots & a_{(k-1)n}\\
a_{(k+1)1} & \cdots & a_{(k+1)(k-1)} & a_{(k+1)(k+1)}^{\prime} & \cdots & a_{(k+1)n}\\
\vdots & {} & \vdots \vdots & {} & \vdots \\
a_{n1} & \cdots & a_{n(k-1)} & a_{n(k+1)} & \cdots & a_{nn}^{\prime}\\
\end{pmatrix} \\
$
Consider the n $\times$ n matrix, \\
$A_\alpha(G_1) =
\begin{pmatrix}
a_{11}^{\prime} & \cdots & a_{1(k-1)} & 0 & a_{1(k+1)} & \cdots & a_{1n} \\
\vdots & {} & \vdots & \vdots & \vdots & {} & \vdots \\
a_{(k-1)1} & \cdots & a_{(k-1)(k-1)}^{\prime} & 0 & a_{(k-1)(k+1)} & \cdots & a_{(k-1)n} \\
0 & \cdots & 0 & 0 & 0 & \cdots & 0 \\
a_{(k+1)1} & \cdots & a_{(k+1)(k-1)} & 0 & a_{(k+1)(k+1)}^{\prime} & \cdots & a_{(k+1)n} \\
\vdots & {} & \vdots & \vdots & \vdots & {} & \vdots \\
a_{n1} & \cdots & a_{n(k-1)} & 0 & a_{n(k+1)} & \cdots & a_{nn}^{\prime} \\
\end{pmatrix} \\
$
which has the same largest eigenvalue as $A_\alpha(G - v_k)$. In fact, all eigenvalues of $A_\alpha(G - v_k)$ are the same as in
$A_\alpha(G_1)$, that possesses an additional zero eigenvalue. 
\par \qquad 
Let B be a diagonal matrix, B = diag(
$\alpha c_1, \alpha c_2,  \cdots, \alpha c_{k-1}, - \alpha d_{k},  \alpha c_{k+1},  \cdots, \alpha c_n ).
$
, then $A_\alpha$ - $A_\alpha(G_1)$ = $a_ke_k^T + e_ka_k^T + B$, where $a_k$ is the column vector $(a_{k1}, a_{k2}, \cdots , a_{kn})^T$ and $e_k$ is the kth basis column vector $(0, 0, \cdots , 0, 1, 0, \cdots , 0)^T$ , where only the kth component is 1. We have \\
\par \quad  $x^T(A_\alpha$ - $A_\alpha(G_1))x$ \\ 
= $x^T (a_k e_k^T + e_k a_k^T + B) x$ \\ 
= $x^T a_k e_k^T x + x^T e_k a_k^T x + x^T B x$ \\
= $2x_k \sum \limits_{i=1}^{n}x_ia_{ki} + \alpha [\sum \limits_{v_i \sim v_k}^{} x_i^2 - d_k x_k^2 ]$ \\
= $2 \rho_\alpha x_k^2 + \alpha [\sum \limits_{v_i \sim v_k}^{} x_i^2 - d_k x_k^2 ]$ \\
Note that the eigenvalue equation written for the component k yields $\sum \limits_{i=1}^{n}x_ia_{ki} = \rho_\alpha x_k$.
\par \qquad 
By Rayleigh principle, we get that  $x^{T}A_\alpha(G_1)x  \leq  \rho_\alpha(G_1)$ for any n $\times$ 1 vector x with
$x^{T}
x = 1$, hence \\
\par \quad  
$\rho_\alpha(G - v_k) \\
= \rho_\alpha(G_1) \\
\geq x^T A_\alpha(G_1) x  \\
= x^T A_\alpha x - x^T(A_\alpha - A_\alpha(G_1))x  \\
= \rho_\alpha - \{2 \rho_\alpha x_k^2 + \alpha [\sum \limits_{v_i \sim v_k}^{} x_i^2 - d_k x_k^2 ]\}
$





\begin{lemma}
\end{lemma}

\begin{lemma}
\end{lemma}












\begin{lemma}

\end{lemma}
















\vskip4mm\noindent{\bf Acknowledgements.}


 The work was partially supported by the National Natural Science Foundation of China under Grants 11771172,12061039.




\begin{thebibliography}{99}









\bibitem{Xu20102}
\textcolor{blue}{ K. Xu, H. Hua, A unified approach to extremal
multiplicative Zagreb indices for trees, unicyclic and bicyclic
graphs, MATCH Commun. Math. Comput. Chem. 68 (2012) 241-256.}

\bibitem{2016Merging}
\textcolor{blue}{
V.~Nikiforov.
\newblock Merging the a- and q-spectral theories.
\newblock {Applicable Analysis \& Discrete Mathematics}, 11(1), 2016.}

%\bibitem{2016Merging}
%\textcolor{blue}{
%V.~Nikiforov.
%\newblock Merging the a- and q-spectral theories.
%\newblock {\em Applicable Analysis \& Discrete Mathematics}, 11(1), 2016.}









\end{thebibliography}

\end{document}
